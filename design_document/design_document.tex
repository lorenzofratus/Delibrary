% Options for packages loaded elsewhere
\PassOptionsToPackage{unicode}{hyperref}
\PassOptionsToPackage{hyphens}{url}
%
\documentclass[10pt, a4page]{report}
\usepackage{amsmath,amssymb}
\usepackage{lmodern}
\usepackage{ifxetex,ifluatex}
\usepackage{float}
\usepackage{caption}
\usepackage{subcaption}
\ifnum 0\ifxetex 1\fi\ifluatex 1\fi=0 % if pdftex
  \usepackage[T1]{fontenc}
  \usepackage[utf8]{inputenc}
  \usepackage{textcomp} % provide euro and other symbols
\else % if luatex or xetex
  \usepackage{unicode-math}
  \defaultfontfeatures{Scale=MatchLowercase}
  \defaultfontfeatures[\rmfamily]{Ligatures=TeX,Scale=1}
\fi
% Use upquote if available, for straight quotes in verbatim environments
\IfFileExists{upquote.sty}{\usepackage{upquote}}{}
\IfFileExists{microtype.sty}{% use microtype if available
  \usepackage[]{microtype}
  \UseMicrotypeSet[protrusion]{basicmath} % disable protrusion for tt fonts
}{}
\makeatletter
\@ifundefined{KOMAClassName}{% if non-KOMA class
  \IfFileExists{parskip.sty}{%
    \usepackage{parskip}
  }{% else
    \setlength{\parindent}{0pt}
    \setlength{\parskip}{6pt plus 2pt minus 1pt}}
}{% if KOMA class
  \KOMAoptions{parskip=half}}
\makeatother
\usepackage{xcolor}
\IfFileExists{xurl.sty}{\usepackage{xurl}}{} % add URL line breaks if available
\IfFileExists{bookmark.sty}{\usepackage{bookmark}}{\usepackage{hyperref}}
\hypersetup{
  hidelinks,
  pdfcreator={LaTeX via pandoc}}
\urlstyle{same} % disable monospaced font for URLs
\usepackage{longtable,booktabs,array}
\usepackage{calc} % for calculating minipage widths
% Correct order of tables after \paragraph or \subparagraph
\usepackage{etoolbox}
\makeatletter
\patchcmd\longtable{\par}{\if@noskipsec\mbox{}\fi\par}{}{}
\makeatother
% Allow footnotes in longtable head/foot
\IfFileExists{footnotehyper.sty}{\usepackage{footnotehyper}}{\usepackage{footnote}}
\makesavenoteenv{longtable}
\usepackage{graphicx}
\makeatletter
\def\maxwidth{\ifdim\Gin@nat@width>\linewidth\linewidth\else\Gin@nat@width\fi}
\def\maxheight{\ifdim\Gin@nat@height>\textheight\textheight\else\Gin@nat@height\fi}
\makeatother
% Scale images if necessary, so that they will not overflow the page
% margins by default, and it is still possible to overwrite the defaults
% using explicit options in \includegraphics[width, height, ...]{}
\setkeys{Gin}{width=\maxwidth,height=\maxheight,keepaspectratio}
% Set default figure placement to htbp
\makeatletter
\def\fps@figure{htbp}
\makeatother
\setlength{\emergencystretch}{3em} % prevent overfull lines
\providecommand{\tightlist}{%
  \setlength{\itemsep}{0pt}\setlength{\parskip}{0pt}}
% \setcounter{secnumdepth}{-\maxdimen} % remove section numbering
\ifluatex
  \usepackage{selnolig}  % disable illegal ligatures
\fi

%Custom titling

\usepackage{titling}

% set up \maketitle to accept a new item
\pretitle{\begin{center}\placetitlepicture\Huge}
\posttitle{\par\lineskip 1em\placesubtitle\placelinklist\end{center}\vskip 3em}
\preauthor{\begin{center}
        \large \lineskip 3em%
        \begin{tabular}[t]{c}}
\postauthor{\end{tabular}\par\placeprofessor\end{center}}
\predate{\begin{center}\large\vskip 3em}
\postdate{\par\placeversion\par\end{center}}

% commands for including the picture
\newcommand{\titlepicture}[2][]{%
  \renewcommand\placetitlepicture{%
    \includegraphics[#1]{#2}\par\medskip
  }%
}
\newcommand{\placetitlepicture}{} % initialization

% commands for including the subtitle
\newcommand{\subtitle}[2][]{%
  \renewcommand\placesubtitle{%
    \Large #2\par\medskip
  }%
}
\newcommand{\placesubtitle}{} % initialization

% commands for including the professor
\newcommand{\professor}[2][]{%
  \renewcommand\placeprofessor{%
    \large Professor: #2\par\medskip
  }%
}
\newcommand{\placeprofessor}{} % initialization

% commands for including the version
\newcommand{\version}[2][]{%
  \renewcommand\placeversion{%
    \large Version: #2\par\medskip
  }%
}
\newcommand{\placeversion}{} % initialization

% commands for including the version
\newcommand{\linklist}[2][]{%
  \renewcommand\placelinklist{%
    \large #2\par\medskip
  }%
}
\newcommand{\placelinklist}{} % initialization
\graphicspath{ {assets/} }

\titlepicture[width=0.75\textwidth]{polimi_logo}
\title{Design Document}
\subtitle{Delibrary}
\author{\href{https://github.com/lorenzofratus}{Lorenzo Fratus} - \href{https://github.com/nicheosala}{Nicolò Sala}}
\professor{Luciano Baresi}
\date{June 22, 2021}
\version{1.0}

\begin{document}

\maketitle

\tableofcontents



\chapter{Introduction}

Delibrary is a mobile application designed to encourage the exchange of physical books.

A user can publish their own books on Delibrary, point out books they would like to own and propose exchanges to other users.
Delibrary is in charge of connecting users interested in a book exchange.

Finally, the idea is to create a "decentralized library", where readers own and directly exchange books.

By putting readers in direct contact, we intend to create new opportunities for socializing and exchanging ideas.

\section{Purpose}

\section{Scope}

\section{Document Structure}



\chapter{Architectural Design}

\section{Backend}
% - MVC pattern: API (Swagger), Services (ExpressJS), database (KnexJS)
% - RESTful APIs using JSON files
% - Heroku and PostgreSQL
% - Swagger for defining API, link to API docs and test

% Ti lascio la descrizione dell'MVC pattern che avevo scritto per l'altro DD, vedi tu se usarla o meno
The Model-View-Controller pattern is used in order to obtain a clear division between the internal representation of information, the way in which that information is presented to the user and the business logic of the system.
This is one of the most common and effective ways to drastically reduce the level of coupling between the various parts of the system.

\section{Frontend}
For the frontend, we chose to implement our application using \emph{Flutter}.\\
Flutter is an open-source framework developed by Google.
It allows you to build multi-platform mobile applications starting from a single codebase (written in \emph{Dart} language).

\subsection{Architectural Patterns}

\subsubsection{MVC Pattern}
Again, we decided to follow the MVC pattern by dividing our system into three decoupled layers.
\begin{itemize}
    \item \textbf{Model}:
          set of custom objects that are used to share data between application components and to communicate with the backend server (through JSON representation);
          the model of our application is immutable, any change in the state will result in the creation of a brand new instance of the object.
    \item \textbf{View}:
          collection of Flutter widgets that are combined to build the user interface, do not contain any logic that directly changes the state of the model.
    \item \textbf{Controller}:
          set of objects that implement communication with the backend server and external service APIs, build and return objects from server responses.
\end{itemize}

\subsubsection{Provider Pattern}
To properly manage the state of our application, we decided to use one of the approaches that are recommended by Google, the Provider pattern.

This pattern is achieved by wrapping the \texttt{MaterialApp} (or any other widget) in a \texttt{ChangeNotifierProvider}.
The provider gives access to a custom object from any widget simply by using the \texttt{BuildContext} object provided during the widget lifecycle.

The object that is \emph{provided} must extend the \texttt{ChangeNotifier} class. In this way, it is possible to use the pre-implemented Observer pattern.\\
When a widget requires access to the provider, it has the possibility to subscribe as a listener and to receive a notification (and automatically rebuild) at any alteration of the state.

The Provider pattern, together with the immutability of the model, allows us to create a sort of \textbf{Single Source of Truth} as any modification to the shared data must be subjected to the provider.
We decided to do this, in order to ensure that any changes correctly update the user interface.

\subsection{External Packages}
As with any framework, Flutter allows us to add external packages that simplify the developement of our code and to implement additional functionalities.
Here we briefly outline the most relevant packages that we introduced and how we used them.
\begin{itemize}
    \item \texttt{provider}:
          used to support the implementation of the Provider pattern, as explained in the previous section.\\
          \underline{\href{https://pub.dev/packages/provider}{Link to \texttt{pub.dev}}}.
    \item \texttt{dio}:
          HTTP client used by controllers to send requests, among other things supports interceptors that are
          used to include the session cookie with any request.\\
          \underline{\href{https://pub.dev/packages/dio}{Link to \texttt{pub.dev}}}.
    \item \texttt{shared\_preferences}:
          allows to use platform-specific persistent storage in a uniform way. It is used to store the session
          cookie to keep sessions active even when the application is closed (one time login).\\
          \underline{\href{https://pub.dev/packages/shared_preferences}{Link to \texttt{pub.dev}}}.
    \item \texttt{layout}:
          component that wraps the \texttt{MaterialApp} widget and helps building flexible layouts by providing
          access to a way to query the screen size. We used it to adapt our UI to bigger screens (tablets).\\
          \underline{\href{https://pub.dev/packages/layout}{Link to \texttt{pub.dev}}}.
    \item \texttt{url\_launcher}:
          allows to launch URLs on multiple platforms. In our specific case it is used to open the default email
          client on the user's device, providing also a precomposed email to contact another user.\\
          \underline{\href{https://pub.dev/packages/url_launcher}{Link to \texttt{pub.dev}}}.
\end{itemize}


\subsection{Application Widgets}
In this section, we discuss the most important widgets we developed in implementing the user interface.
They are listed following the folder structure of our application.

\subsubsection{Components}
Independent widgets that can be used as building blocks inside a \texttt{Scaffold}.
All of them accept attributes to customize their behavior, some of them even accept other widgets as children.

\begin{itemize}
    \item \texttt{cards/book-card}:
          takes a \texttt{Book} object as input and displays its information on a card.
          It can also show a smaller version of the card by setting the \texttt{preview} attribute to \texttt{true}.
          It is used in any occasion in which a book needs to be rendered.
    \item \texttt{cards/exchange-card}:
          dual of the \texttt{cards/book-card} widget but for an \texttt{Exchange} object.
    \item \texttt{cards/item-card-list}:
          takes either a \texttt{BookList} or an \texttt{ExchangeList} object as input and displays the list of cards.
          It can be used to perform lazy loading by providing a \texttt{nextPage} callback that is called when the user
          approaches the end of the list.
    \item \texttt{modals/draggable-modal-page}:
          shorthand used to customize the existing \texttt{DraggableScrollableSheet} widget.
          It takes a \texttt{builder} callback as input to pass the child element.
          It is used every time this type of modal is rendered in the application, such as in the info pages.
    \item \texttt{utils/padded-grid}:
          general purpose widget, builds a \texttt{CustomScrollView} that arranges children in a grid with
          dynamic layout (depends on the screen size). It is used to provide a flexible interface.
\end{itemize}

\subsubsection{Routes}
Widgets representing a page of our application.
They contain a \texttt{Scaffold} and most of them (excluding info pages) are \emph{named} in order to be easily accessed from anywhere.

\begin{itemize}
    \item \texttt{info-pages/book-info}:
          takes a \texttt{Book} object as input and displays all its information.
          The page includes a list of buttons that allow the user to perform different actions, these actions are computed dynamically based on the \texttt{Book} object received as input.
          For example, if the book is included in the user's library, they could either choose to \emph{remove it} or to \emph{move it to the wishlist}.
          It is shown anytime the user taps on a \texttt{book-card}.
    \item \texttt{info-pages/exchange-info}:
          dual of the \texttt{info-pages/book-info} widget but for an \texttt{Exchange} object.
    \item \texttt{home}:
          main page of the application, composed by an \texttt{IndexedStack} containing 4 pages (see Screens section) that can be switched using a \texttt{BottomNavigationBar}.
    \item \texttt{login}:
          login page, this is the first that the user encounters when opening the application (if not already logged in) or after logout.
    \item \texttt{register}:
          registration page, reachable from the login page when clicking on the proper link.
    \item \texttt{global-search}:
          page containing the \emph{global search bar} that is used to find a book by title, author name or ISBN through the Google Books API.
    \item \texttt{archived}:
          page in which the user can see the exchanges that have been archived (refused or completed).
\end{itemize}

\subsubsection{Screens}
Widgets representing a page of our application.
Unlike the widgets presented in the Routes section, they do not contain a \texttt{Scaffold} as they are enclosed inside the \texttt{home} page.

\begin{itemize}
    \item \texttt{position-search}:
          first page that the user sees after the login, like \texttt{global-search} presents a search bar that is used to find a book
          in the Delibrary community, filtering by Province and Town.
    \item \texttt{library}:
          page in which the user can see and manage the books that are in their library or wishlist.
          From there, they can go to the \texttt{global-search} page using the proper button.
    \item \texttt{exchanges}:
          page in which the user can see and manage the exchanges proposed by other users or involving the books in their library.
          From there, they can go to the \texttt{archived} page using the proper button.
    \item \texttt{profile}:
          page displaying the profile of the user. From there they can edit their data (except the username), change their password, or log out.
\end{itemize}

\chapter{External Services}

\section{Google Books API}

\section{Comuni ITA API}



\chapter{User Interface Design}

\section{UX Diagram}

\section{Screenshots}



\chapter{Implementation, Integration and Test}

\section{Integration History}
% 1. API definition
% 2. first backend implementation && first frontend implementation and comunication with Google Books API
% 3. library management
% 4. exchange management
% 5. INTANTO I TEST ...

\section{Backend Tests}

\section{Frontend Tests}



\chapter{Possible Extensions and Updates}
% Search in Delibrary by title, author name or ISBN
% Filtering system (author, genre, ...)
% Secure login, registration and authentication



\end{document}